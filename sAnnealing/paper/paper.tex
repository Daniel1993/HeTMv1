\documentclass[a4paper]{article}
%
\usepackage{amsmath}
\usepackage{amsthm}
\usepackage{url}
\usepackage{adjustbox}
\usepackage{algpseudocode}
\usepackage{algorithm}
\usepackage{pstricks,pst-node}
\usepackage[square,comma,sort]{natbib}

\bibliographystyle{unsrtnat}

\newtheorem{definition}{Definition}

\DeclareMathOperator{\inn}{in}
\DeclareMathOperator{\out}{out}

% STUFF TODO:
%
% Choose a font for Max-DAG problem

\newtheorem{theorem}{Theorem}
\newtheorem{lemma}{Lemma}
\newtheorem{corolary}{Corolary}

\author{Lu\'is M. S. Russo\\
  \texttt{luis.russo@tecnico.ulisboa.pt}
  \and
  Daniel Castro\\
 \texttt{daniel.castro@ist.utl.pt}
  \and
  Aleksandar Ilic\\
  \texttt{aleksandar.ilic@inesc-id.pt}
  \and
  Paolo Romano\\
  \texttt{romano@inesc-id.pt}
}

%(1000 readers)
\title{}

%\date{\today}
\date{}

\begin{document}

\maketitle

\begin{abstract}
%(4 sentences, 100 readers)
% 1.  State the problem
% 2.  Say why it’s an interesting problem
% 3.  Say what your solution achieves
% 4.  Say what follows from your solution

\end{abstract}
%

\section{Introduction}
\label{sec:introduction}
%(1 page, 100 readers)

\section{The problem}
\label{sec:problem}
%(1 page, 10 readers)

\begin{figure}[tb]
  \begin{center}
    \begin{pspicture}(0.0,-4.0)(7.5,2.5)
      %
      \rput[bl](0,0){
        \psframe[framearc=0.5](-0.3,-0.3)(7.8,3.0)
        % 
        \rput[bl](0,0){
          \psmatrix[mnode=circle,colsep=0.8,rowsep=0.5]
          & a \\
          b & c & d & e & f & g \\
          &   & h & i & j
          \endpsmatrix
          % 
          \psset{arrowsize=0.15}
          % 

          % SCC
          \ncline{->}{1,2}{2,2}
          \ncline{->}{2,2}{2,1}
          \ncline{->}{2,1}{1,2}

          % SCC
          \ncline{->}{2,3}{2,4}
          \ncline{->}{2,4}{3,3}
          \ncline{->}{3,3}{2,3}
          \ncline{->}{3,3}{3,4}
          \ncline{->}{3,4}{2,4}

          \psset{arcangle=30}
          % SCC
          \ncarc{->}{2,5}{3,5}
          \ncarc{->}{3,5}{2,5}

          % Inter connections
          \ncline{->}{1,2}{2,3}
          \ncline{->}{1,2}{2,5}

          \ncline{->}{2,2}{2,3}

          \ncline{->}{2,4}{2,5}
          \ncline{->}{3,4}{3,5}

          \ncline{->}{2,5}{2,6}
          % 
        }
      }
      %
      % Now the solution
      %
      \rput[bl](0,-3.5){
        \psframe[framearc=0.5](-0.3,-0.3)(7.8,3.0)
        % 
        \rput[bl](0,0){
          \psmatrix[mnode=circle,colsep=0.8,rowsep=0.5]
          & a \\
          b &   & d &   &  & g \\
          &   & h & i & j
          \endpsmatrix
          % 
          \psset{arrowsize=0.15}
          % 

          % SCC
%          \ncline{->}{1,2}{2,2}
%          \ncline{->}{2,2}{2,1}
          \ncline{->}{2,1}{1,2}

          % SCC
%          \ncline{->}{2,3}{2,4}
%          \ncline{->}{2,4}{3,3}
          \ncline{->}{3,3}{2,3}
          \ncline{->}{3,3}{3,4}
 %         \ncline{->}{3,4}{2,4}

          \psset{arcangle=30}
          % SCC
%          \ncarc{->}{2,5}{3,5}
%          \ncarc{->}{3,5}{2,5}

          % Inter connections
          \ncline{->}{1,2}{2,3}
%          \ncline{->}{1,2}{2,5}

%          \ncline{->}{2,2}{2,3}

%          \ncline{->}{2,4}{2,5}
          \ncline{->}{3,4}{3,5}

%          \ncline{->}{2,5}{2,6}
          % 
        }
      }
    \end{pspicture}
  \end{center}
  %
  \caption{Illustration of the max-DAG problem}
  \label{fig:problem}
\end{figure}

\section{Simulated Annealing}
\label{sec:my-idea}
%(2 pages, 10 readers)
%
We propose a simple simulated annealing algorithm to determine the maximum
sub-DAG. A state consists in a sub-DAG. This DAG corresponds to the
sub-graph induced by a set $V'$ of vertexes, that is a sub-set of the set
of vertexes $V$ of the original graph $G_D = (V, E_D)$. At each step a
candidate vertex $v$ is chosen. The candidate is chosen from $V \setminus
V'$. The goal is to add $v$ to $V'$. Hence we need to test whether the
sub-graph induced by $V' \cup \{v\}$ still is a DAG. If so $v$ can be
safely added to $V'$, otherwise $v$ is part of at least one cycle in the
induced sub-graph.

If it is safe to add $v$ to $V'$ this process can be executed without
further decisions. Otherwise a harder decision presents itself, either $v$
is rejected, which means that it is not added to $V'$, or other vertexes of
$V'$ need to be removed so that $v$ can be accepted. This decision depends
on how many vertexes need to be removed from $V'$. Hence we determine a
sub-set $V'' \subseteq V'$ such that sub-graph induced by $V'' \cup \{v\}$
is a DAG. Ideally $V''$ should be as large as possible. Assume for now that
we can obtain such a set $V''$. The choice the simulated annealing
procedure must ponder is whether $v$ gets rejected and $V'$ is kept intact
or whether $v$ is accepted, in which case $V'$ is replaced by $V'' \cup
\{v\}$. The simulated annealing decides by considering the value $\Delta
E=1-|V' \setminus V''|$. In each step the algorithm is allowed a limit
$\ell$ that specifies how much energy the current optima can loose. If
$\ell \leq \Delta E$ then the modification is allowed, meaning that the
candidate $v$ is accepted, even though this decreases the current optimum
value. Otherwise candidate $v$ is rejected. The value $\ell$ is determined
by the current temperature value $t$ and a random number, chosen uniformly
from the interval $[0,1]$. A precise description is given in
Section~\ref{sec:related-work}.

In this section we prefer to focus on how to effectively determine
$V''$. This is again an instance of our maximum sub-DAG problem. Even
though we can execute a depth first search, or Tarjan's algorithm, to
identify which vertexes are involved in cycles it is not immediately clear
which ones should be removed. Hence we use an heuristic approach, which
moreover is faster than the $O(|V|+|E_D|)$ time bound required by a
DFS. This is a very expensive requirement for a single decision of the
algorithm. Our heuristic requires $O(V \log V)$ time, more precisely
$O(d(v) \log V)$ time, where $d(v)$ represents the degree of the candidate
node\footnote{In directed graphs there is both in-degree and out-degree,
  respectively the number of edges $(u,v) \in E_D$ and $(v,u) \in E_D$ for
  some $u$. By degree we mean the sum of both these values.}.  Our decision
algorithm can actually be even faster, the $O(d(v) \log V)$ bound is for
the worst case.

However this decision is not precise, meaning that it underestimates the
set $V''$. It however has the crucial guarantee that the sub-graph induced
by $V'' \cup \{v\}$ is a DAG. This imprecision means that the annealing
procedure will reject candidates that could have been accepted. We use
several approaches to mitigate this consequence. The simplest method is to
choose a higher temperature parameter, as this gives more slack to vertexes
that would otherwise be rejected. A more efficient method is to reorganize
the current DAG so that the decision becomes more precise. This is a costly
operation and improving the precision for a node $v$ might decrease the
precision for other nodes. Hence we use this process sparingly.

Our procedure is the following. Since the sub-graph induced by $V'$ is a
DAG we keep it sorted, by some topological ordering, i.e.,
$s:V' \mapsto \{0, \ldots, |V'|-1\}$ is a bijection. Whenever a candidate
node $v$ is being evaluated we determine the positions of the incoming
edges and outgoing edges. Precisely we determine the maximum position of
the incoming edges and the minimum position of the outgoing edges as
follows:
%
 \begin{itemize}
 \item $\max(v) = \max \{s(u)\ |\ (u, v) \in E_D\}$
 \item $\min(v) = \min \{s(u)\ |\ (v, u) \in E_D\}$
 \end{itemize}
 %
 %
 Whenever $\max(v) < \min(v)$ the candidate vertex $v$ is safe, because $v$
 is not involved in any cycle in $V' \cup \{v\}$. Hence $v$ can be safely
 added to $V'$. On the other hand when $\max(v) \geq \min(v)$ it may be the
 case that $v$ is part of some cycle. We do not exactly determine whether
 it actually is part of some cycle. Instead we focus on identifying a set
 of incompatible nodes $I(v)$ such that in $V' \setminus I(v)$ the
 corresponding values $\max(v)$ and $\min(v)$ respect the desired
 inequality that $\max(v) < \min(v)$.

 A simple choice for $I(v)$ are the vertexes $u$ of $V'$ for which
 $\min(v) \leq s(u) \leq \max(v)$ and $(v,u) \in E_D$. In Section~\ref{} we
 consider a more general processes to obtain $I(v)$, for simplicity we
 will, for now, restrict our attention to this choice.
 Figure~\ref{fig:selection} illustrates this choice. The edges linking
 vertexes in $V'$ are drawn with dashed gray lines. The edges between nodes
 of $V'$ and the candidate node $v$ are drawn with solid black lines. The
 vertexes are labeled directly with the $s(v)$
 values. Figure~\ref{fig:selection} illustrates three examples. In the top
 example we have that $\max(v) = 2 < 3 = \min(v)$. In this example it is
 safe to add $v$ to $V'$. In the middle example we have that
 $\min(v) = 2 \leq 4 = \max(v)$. In this case the annealing procedure needs
 to make a choice. We have that $I(v) = \{2, 3\}$, because these nodes are
 less than or equal to $4 = \max(v)$ and the edges $(v, 2)$ and $(v, 3)$
 are part of the graph. If the choice is favorable to $v$ nodes $2$ and $3$
 must be removed before $v$ is added. This consequence is highlighted by
 the gray background of these nodes. The bottom case is almost identical to
 the middle case. The decision and consequences are the same. However in
 this example removing $2$ and $3$ was not actually necessary. Note that
 node $4$ is not restricted by any of edges in the sub-graph and might just
 as well be located before node $0$ in the topological order, in which case
 the candidate node $v$ would have no problems being accepted and its
 acceptance would entail no removals. However our selection procedure is
 not able to reach this conclusion, instead it behaves as in the middle
 case. This is a necessary trade-off for the fast performance of the
 decision heuristic we propose. We will now explain how to implement this
 heuristic efficiently and also how to mitigate its imprecise decisions.
%
 \begin{figure}[tb]
   \begin{center}
%     \begin{pspicture}(0,-12)(13,2)
     \begin{pspicture}(0,-9)(9.75,1.5)
     \psscalebox{0.75}{
       %
       \psset{arrowsize=0.25}
       %
       \rput(0.5,0.0){
         %
         \psframe[framearc=0.5](-0.5,-2.0)(12.5,2.5)
         %
         \cnodeput(0,0){0}{0}
         \cnodeput(2,0){1}{1}
         \cnodeput(4,0){2}{2}
         \cnodeput(6,0){3}{3}
         \cnodeput(8,0){4}{4}
         \cnodeput(10,0){5}{5}
         \cnodeput(12,0){6}{6}
         \nput[rot=0,labelsep=0]{50}{2}{$\max(v)$}
         \nput[rot=40]{60}{3}{$\min(v)$}
         %
         \psset{linecolor=black!70}
         \psset{linestyle=dashed}
         %
         \ncline{->}{0}{1}
         \ncline{->}{1}{2}
         \ncline{->}{2}{3}
         \ncline{->}{3}{4}
         \ncline{->}{4}{5}
         \ncline{->}{5}{6}
         %
         \psset{arcangle=30}
         \ncarc[arcangle=40]{->}{0}{6}
         \ncarc{->}{0}{2}
         \ncarc[arcangle=40]{->}{3}{6}
         \ncarc{->}{3}{5}
         %
         \psset{linecolor=black}
         \psset{linestyle=solid}
         \cnodeput(5,-1.5){v}{$v$}
         %
         \ncline{->}{0}{v}
         \ncline{->}{2}{v}
         \ncline{->}{v}{3}
         \ncline{->}{v}{4}
         \ncline{->}{v}{6}
       }
       %
       \rput(0.5,-5.0){
         %
         \psframe[framearc=0.5](-0.5,-2.0)(12.5,2.5)
         %
         \cnodeput(0,0){0}{0}
         \cnodeput(2,0){1}{1}
         \cnodeput[fillcolor=black!25,fillstyle=solid](4,0){2}{2}
         \cnodeput[fillcolor=black!25,fillstyle=solid](6,0){3}{3}
         \cnodeput(8,0){4}{4}
         \cnodeput(10,0){5}{5}
         \cnodeput(12,0){6}{6}
         \nput[rot=0,labelsep=0]{50}{2}{$\min(v)$}
         \nput[rot=0,labelsep=0]{-60}{4}{$\max(v)$}
         %
         \psset{linecolor=black!70}
         \psset{linestyle=dashed}
         %
         \ncline{->}{0}{1}
         \ncline{->}{1}{2}
         \ncline{->}{2}{3}
         \ncline{->}{3}{4}
         \ncline{->}{4}{5}
         \ncline{->}{5}{6}
         %
         \psset{arcangle=30}
         \ncarc[arcangle=40]{->}{0}{6}
         \ncarc{->}{0}{2}
         \ncarc[arcangle=40]{->}{3}{6}
         \ncarc{->}{3}{5}
         %
         \psset{linecolor=black}
         \psset{linestyle=solid}
         \cnodeput(5,-1.5){v}{$v$}
         %
         \ncline{->}{0}{v}
         \ncline{->}{v}{2}
         \ncline{->}{v}{3}
         \ncline{->}{4}{v}
         \ncline{->}{v}{6}
       }
       %
       \rput(0.5,-10.0){
         %
         \psframe[framearc=0.5](-0.5,-2.0)(12.5,2.5)
         %
         \cnodeput(0,0){0}{0}
         \cnodeput(2,0){1}{1}
         \cnodeput[fillcolor=black!25,fillstyle=solid](4,0){2}{2}
         \cnodeput[fillcolor=black!25,fillstyle=solid](6,0){3}{3}
         \cnodeput(8,0){4}{4}
         \cnodeput(10,0){5}{5}
         \cnodeput(12,0){6}{6}
         \nput[rot=0,labelsep=0]{50}{2}{$\min(v)$}
         \nput[rot=0,labelsep=0]{-60}{4}{$\max(v)$}
         %
         \psset{linecolor=black!70}
         \psset{linestyle=dashed}
         %
         \ncline{->}{0}{1}
         \ncline{->}{1}{2}
         \ncline{->}{2}{3}
         % \ncline{->}{3}{4}
         \ncline{->}{4}{5}
         \ncline{->}{5}{6}
         %
         \psset{arcangle=30}
         \ncarc[arcangle=40]{->}{0}{6}
         \ncarc{->}{0}{2}
         \ncarc[arcangle=40]{->}{3}{6}
         \ncarc{->}{3}{5}
         %
         \psset{linecolor=black}
         \psset{linestyle=solid}
         \cnodeput(5,-1.5){v}{$v$}
         %
         \ncline{->}{0}{v}
         \ncline{->}{v}{2}
         \ncline{->}{v}{3}
         \ncline{->}{4}{v}
         \ncline{->}{v}{6}
       }
       %
     }
     \end{pspicture}
   \end{center}
   \caption{Illustration of vertex evaluation.}
   \label{fig:selection}
 \end{figure}
 %
%
\section{The details}
\label{sec:details}
%(5 pages, 3 readers)
%
In this section we will explain how to efficient implement the decision
heuristic described in Section~\ref{sec:my-idea} and also a global
re-organizing procedure that mitigates the limitations of the decision
heuristic. However before we explain these procedures it is important to
establish some results related to the nature of the problem we are
considering. In particular we will establish that the problem is NP-Hard
and Poly-APX-complete. This validates our simulated annealing approach, as
it is unlikely that there exist polynomial time algorithms that can solve
or approximate this problem, with a constant or poly logarithmic
approximation ratio.
%

\subsection{Complexity Analysis}
\label{sec:complexity-analysis}
%
\begin{figure}[tb]
  \label{fig:reduction}
  \begin{center}
    %
    \begin{pspicture}(9,4)
      %
      \rput(0.5,0.5){
        %
        %
        \cnodeput(0,0){1}{$1$}
        \cnodeput(3,0){2}{$2$}
        \cnodeput(3,3){3}{$3$}
        \cnodeput(0,3){4}{$4$}
        \cnodeput(1,2){5}{$5$}
        \cnodeput(1,1){6}{$6$}
        \cnodeput(2,1){7}{$7$}
        \cnodeput(2,2){8}{$8$}
        %
        \ncline{-}{1}{2}
        \ncline{-}{2}{3}
        \ncline{-}{3}{4}
        \ncline{-}{4}{1}
        %
        \ncline{-}{5}{6}
        \ncline{-}{6}{7}
        \ncline{-}{7}{8}
        \ncline{-}{8}{5}
        %
        \ncline{-}{1}{6}
        \ncline{-}{4}{5}
        \ncline{-}{3}{8}
        \ncline{-}{7}{2}
      }

      \rput(5.5,0.5){
        %
        \psset{arrowsize=0.15}
        \psset{arcangle=15}
        %
        \cnodeput(0,0){1}{$1$}
        \cnodeput(3,0){2}{$2$}
        \cnodeput(3,3){3}{$3$}
        \cnodeput(0,3){4}{$4$}
        \cnodeput(1,2){5}{$5$}
        \cnodeput(1,1){6}{$6$}
        \cnodeput(2,1){7}{$7$}
        \cnodeput(2,2){8}{$8$}
        %
        \ncarc{->}{1}{2}
        \ncarc{->}{2}{1}
        \ncarc{->}{2}{3}
        \ncarc{->}{3}{2}
        \ncarc{->}{3}{4}
        \ncarc{->}{4}{3}
        \ncarc{->}{4}{1}
        \ncarc{->}{1}{4}
        %
        \ncarc{->}{5}{6}
        \ncarc{->}{6}{5}
        \ncarc{->}{6}{7}
        \ncarc{->}{7}{6}
        \ncarc{->}{7}{8}
        \ncarc{->}{8}{7}
        \ncarc{->}{8}{5}
        \ncarc{->}{5}{8}
        %
        \ncarc{->}{1}{6}
        \ncarc{->}{6}{1}
        \ncarc{->}{4}{5}
        \ncarc{->}{5}{4}
        \ncarc{->}{3}{8}
        \ncarc{->}{8}{3}
        \ncarc{->}{7}{2}
        \ncarc{->}{2}{7}
      }
      %
    \end{pspicture}
    %
  \end{center}
  \caption{}
  \label{fig:reduction}
\end{figure}

First we show that the problem at hand is NP-complete. In this case we
consider the sub-DAG problem, which given a graph $G_D = (V, E_D)$ and an
integer $k$ consists in determining if there is a set of vertexes $V'$ of
size $k$ such that the sub-graph induced by $V'$ is a DAG.
%
\begin{theorem}
  The Sub-DAG problem is NP-complete.
\label{teo:NPC}
\end{theorem}
%
\begin{proof}
  First we show that the Sub-DAG $\in$ NP. For a given graph
  $G = (V, E_D)$, we consider a sub-set $V' \subseteq V$ of vertices as a
  certificate. First we check if $V'$ contains exactly $k$ elements, where
  $k$ is the desired number of vertexes specified by the Sub-DAG instance
  at hand. This requires only $O(V')$ if $V'$ is given as a linked list of
  $O(V)$ if it is given as an array of booleans. Verifying that the induced
  sub-graph is a DAG amounts to checking that it does not contain
  cycles. This can be solved by executing a DFS on the sub-graph and
  checking that it does not contain back edges. The resulting algorithm
  requires $O(V'+E_D)$ time, when the structure that identifies the sub-set
  $V'$ takes $O(1)$ time to determine if a vertex $v$ is in $V'$ or not. An
  array of booleans indexed over $V$ suffices. Hence in total a
  verification algorithm requires $O(V + E)$ time and space, which is
  polynomial in the size of the input.

  Next we show that Max-DAG is NP-Hard. For this goal we use a
  reduction from the Independent-Set problem. Given an undirected graph
  $G = (V, E)$ the Independent-Set problem consists in determining whether
  there exists a set of vertexes $V' \subseteq V$, of size $k$, such that
  the induced sub-graph contains no edges, i.e., for any $\{u, v\} \in E$
  it can not be that both the vertexes exist in $V'$, i.e., either
  $u \notin V'$ or $v \notin V'$. Given an instance graph $G=(V, E)$, of
  the Independent-Set problem, we need to build an instance
  $G_D = (V_D, E_D)$ of the Sub-DAG problem. We illustrate this reduction
  reduction in Figure~\ref{fig:reduction}. The given undirected graph $G$
  is shown on the left. The set of vertexes $\{1,3,5,7\}$ is an example of
  an independent set in this graph.

  For this process we need to construct $G_D$ to be a representation of $G$
  as a directed graph. In essence for every edge $\{u,v\} \in E$ we will
  add two edges $(u,v)$ and $(v,u)$ to $E_D$. The set of vertexes is kept
  unaltered, i.e., $V_D=V$.  We illustrate this construction in
  Figure~\ref{fig:reduction}. The graph in the right is directed, which is
  indicated by the arrows at the end of the edges. This construction yields
  $|V_D| = |V|$ and $|E_D| = 2 |E|$ and therefore can be obtained in linear
  time and space.

  To finish our reduction we require that the sub-DAG to determine in $G_D$
  should also have size $k$. Hence to finish this reduction we claim that
  it is possible to find an independent set of size $k$ for $G$ if and only
  if it is possible to find a sub-DAG with $k$ vertexes in $G_D$.

  We start by showing that if there is an independent set of size $k$
  in $G$ then there is a sub-DAG with $k$ vertexes in $G_D$. The reduction
  is trivial. Given that $V'$ is an independent set of $G$ then $V'$ is
  also a DAG in $G_D$, because the sub-graph induced by $V'$ in $G_D$
  contains no edges. Assume by contradiction that $(u,v) \in E_D$ is an
  edge in the sub-graph induced by $V'$, then $u \in V'$ and $v \in
  V'$. Hence by construction $\{u,v\}$ must also be an edge in $E$, which
  is a contradiction because $V'$ should be an independent set but it
  contains $u$ and $v$ and the edge $\{u,v\}$ exists in $E$.

  Finally we need to show the if we obtain some set of vertexes
  $V'\subseteq V_D$ such that the induced sub-graph of $G_D$ is a DAG then
  the set $V'$ is an independent set in $G$. Let $\{u, v\} \in E$ be an
  edge of $E$ then $(u,v)$ and $(v,u)$ are edges of $E_D$, which form a
  cycle. Therefore either $u \notin V'$ or $v \notin V'$, otherwise the DAG
  contained a cycle. Hence $V'$ is an independent set.
%
\end{proof}
%
This complexity condition seems to imply that a pragmatic approach to the
problem either consists in studying exponential algorithms or in polynomial
time approximation algorithms. On closer inspection however even this
second hypothesis is infeasible, unless P=NP. This is a two fold
consequence. On the one hand the Max Independent Set is
Poly-APX-complete. On the other hand the reduction we gave is extremely
robust and therefore qualifies as an approximation preserving reduction.
Specifically it qualifies as an S-reduction. The strictest of several
alternatives, which more over preserve several membership conditions,
see~\citet{Crescenzi}. The approximation ratio is equal in the Max
Independent Set and in max-DAG, because it is the exactly the same set
sub-set $V'$ from the same set of vertexes $V$, only the set of edges is
modified.
%
\begin{corolary}
The max-DAG problem is Poly-APX-complete under PTAS-reductions.
  \label{cor:APX}
\end{corolary}
%
\begin{proof}
  Clearly max-DAG is in Poly-APX. An algorithm which simply chooses one
  vertex $v$ of $V$ runs in polynomial time and is a $|V|$ approximation
  to max-DAG, because the maximum DAG must be a sub-set $V'$ of $V$.

  To show that max-DAG is Poly-APX-complete we note that the Max
  Independent Set is Poly-APX-complete under PTAS-reductions,
  see~\citet*{Bazgan_2005}. Hence we use the reduction in
  Theorem~\ref{teo:NPC} to reduce Max Independent Set to max-DAG. Note that
  S-reductions are also PTAS-reductions. Because the composition of a
  PTAS-reduction and an S-reduction still is a PTAS-reduction, we only need
  to confirm that the reduction in Theorem~\ref{teo:NPC} is an S-reduction.
  Using the notation of~\citet{Crescenzi} we need to describe a function
  $g$ that transforms solutions of max-DAG into solutions of Max
  Independent Set, in polynomial time, and preserves the solution
  measure. Formally $m_A(x, g(x,y)) = m_B(f(x),y)$. This is trivial because
  a set of vertexes $V'$ that induces a DAG as a sub-graph of the
  transformed instance $G_D = f(x)$ is also an independent set in the
  original graph $G=x$, as shown in the proof of
  Theorem~\ref{teo:NPC}. Hence $g(x,y) = V'$, for $y = V'$. Moreover the
  measure values in these problems are the number of vertexes in the set
  $V'$. Therefore the necessary equality simplifies to
  $m_A(x,g(x,y)) = |V'| = m_B(f(x),y)$.
\end{proof}
%
Even thought it is unlikely that Max Independent Set can be approximated to
ratio better than a polynomial in polynomial time. Polynomial time
approximation algorithms do exit for bounded degree graphs. In particular a
simple greedy algorithm obtains a $(\delta + 2)/3$ approximation ratio, for
a graph with maximum degree $\delta$. Moreover the same algorithm is also
guaranteed an approximation ratio of $(d+2)/2$, where $d$ is the average
degree of the graph. See~\citet*{Halld_rsson_1997}. Hence this algorithm
strongly influences our approach to the max-DAG problem.

The algorithm works as follows: we start with an initially empty
independent set $V' = \emptyset$ and an instance graph $G$. At each set we
choose the vertex $v$ with minimum degree, i.e.,
$d(v) = \min\{d(v')\ |\ v' \in V\}$, where $d(v)$ represents the degree of
$v$. By degree we mean the size of its neighbor set
$N(v) = \{u\ |\ \{v, u\} \in E\}$. The vertex $v$ is added to $V'$ and
removed, along with its neighbors, from the set $V$. Precisely set
$V' \leftarrow V' \cup \{v\}$ and
$V \leftarrow V \setminus (\{v\} \cup N(v))$. Likewise the set $E$ is
cleaned from edges that connect vertexes that no longer exist in $V$. The
algorithm terminates when the set $V$ becomes empty. Note that at each step
the set $V'$ is an independent set.

Consider the following simulated annealing approach algorithm, which is
similar in spirit to one we use for max-DAG and to the greedy algorithm we
have just described. We also start with an empty set $V'$, but we keep the
graph $G$ unmodified. At each step a vertex $v$ is allowed into $V'$ if
there is no edge $\{u,v\} \in E$ such that $u \in V'$. Such a vertex $u$ is
incompatible with $v$. On the other hand if such incompatible vertexes $u$
do exist the annealing algorithm counts how many of them exist. If this
number is not too big and the temperature and the random choice allow for
it the vertex $v$ may be allowed into $V'$ anyway. In which case all the
incompatible vertexes $u$ first need to be removed from $V'$. Note that
like in the greedy algorithm the degree of a node $v$ plays an important
role on whether it is allowed in the solution or note. A node $v$ for which
$d(v)$ is big is unlikely to be allowed into $V'$ because it is likely to
have incompatible nodes in $V'$. Even if a node with a high degree is
allowed into $V'$ at some step it will be pressured out because all its
neighbors are incompatible with it.

%
% \begin{figure}[tb]
%   \begin{center}
%     \begin{pspicture}(0,-3)(7,4)
%       %
%       \psset{arrowsize=0.25}
%       \rput(0.5,0.5){
%         %
%         %
%         \cnodeput(0,0){1}{$A$}
%         \cnodeput(3,0){2}{$B$}
%         \cnodeput(3,3){3}{$C$}
%         \cnodeput(0,3){4}{$D$}
%         %
%         \ncline{->}{1}{2}
%         \ncline{->}{2}{4}
%         \ncline{->}{3}{2}
%         \ncline{->}{4}{3}
%         \ncline{->}{4}{1}
%       }
%       %
%       \rput(0.5,-2.0){
%         %
%         \cnodeput(0,0){1}{$A$}
%         \cnodeput(2,0){2}{$B$}
%         \cnodeput(6,0){3}{$C$}
%         \cnodeput(4,0){4}{$D$}
%         %
%         \nput{90}{1}{1}
%         \nput{90}{2}{2}
%         \nput{-90}{4}{2}
%         \nput{-90}{3}{1}
%         %
%         \ncline{->}{1}{2}
%         \ncline{->}{2}{4}
%         \ncline{->}{4}{3}
%         %
%         %
%         \psset{linestyle=dashed}
%         \psset{linecolor=black!70}
%         \ncarc[arcangle=30]{->}{4}{1}
%         \ncarc[arcangle=-30]{->}{3}{2}
%       }
%       %
%     \end{pspicture}
%   \end{center}
%   \caption{}
%   \label{fig:DFS}
% \end{figure}
%

\subsection{DAG Data Structures}
\label{sec:dag-data-structures}
%
Given that we now have a rough description of the simulated annealing
procedure we can discuss the data structures used to implement the decision
heuristic described in Section~\ref{sec:my-idea}. As explained in that
Section we aim to maintain a representation of the set $V'$ of vertexes,
such that the induced sub-graph is a DAG. As the annealing algorithm
proceeds vertexes get added and removed from $V'$. Moreover for our
heuristic decision it is necessary to keep this set ordered, by some
topological order of the DAG. Given these constraints we choose to store the
vertexes of $V'$ in a binary search tree, in particular we used splay trees
in our implementation. Any binary search tree provides the operations
necessary for our algorithm, but a BST with worst case $O(\log V)$
guarantees would be ideal, such as AVL's or Red-Black trees. Splay trees
might suffer from occasional worst case bad performance, but they provide
amortized time guarantees of several desirable time bounds. Given that our
algorithm performs a sequence of operations the amortized time bounds are
sufficient. Single operation worst case guarantees are not crucial.
Storing the sequence of nodes in a BST has the added advantage that
whenever a node is added or removed the new positions of the nodes are
obtained automatically. Consider again the example at the top of
Figure~\ref{fig:selection}. In this case we have that
$\max(v) = 2 < 3 = \min(v)$ and therefore it is safe to add the node $v$ to
the set $V''$. This consists in inserting the vertex $v$ into the splay
tree so that it becomes the new vertex at position $3$. Hence the previous
vertex at position $3$ becomes the vertex at position $4$, the previous
vertex at position $4$ becomes the vertex at position $5$ and so on. This
is obtained with the insertion operation in $O(\log n)$ amortized time in
splay trees. To determine the current position of a vertex it is enough to
store the current sub-tree size in each node of the splay tree.

Likewise whenever a node needs to be removed from the set $V'$ it is removed
from the splay tree. Consider for example the cases in the middle and bottom
of Figure~\ref{fig:selection}. In these cases the nodes at position $2$ and
$3$ need to be removed from the splay tree. After this operation, and
before the node $v$ is inserted, the node that was at position $4$ becomes
the node that is at position $2$, the node that was at position $5$ becomes
the node that is at position $3$ and so on.

We can now focus on the exact procedure we use for our decision
heuristic. According to the description in Section~\ref{sec:my-idea} we
need to determine $\min(v)$ and $\max(v)$. To determine $\min(v)$ we need
to visit the neighborhood of $v$, i.e., all the vertexes $u \in V'$ such
that there exits an edge $(v, u) \in E_D$. Actually we only need the
neighborhood in the sub-graph induced by $V'$. However, because our graph
$G_D$ is represented in an adjacency list, we actually end up traversing
the complete neighborhood of $v$ to obtain the necessary nodes $u$. For
each node $u$ that is a neighbor a of $v$, i.e., for which a node $(v,u)$
exists we first check if $u \in V'$, by checking if $u$ is currently in the
splay tree. If so, we then determine its position in the topological order.
This is actually the size of the left sub-tree of node $u$, since looking
up node $u$ involves a splay operation that pulls it to the root of the
tree. In fact looking up $u$ is slightly tricky, because the splay tree is
ordered by the topological order. Therefore this process is not computed as
a search over the tree, instead we keep an array that for each node
$u \in V$ keeps the information of whether the node is in the tree or not,
and if so where is it in the tree. To be absolutely precise each node of
the splay tree is represented by a \texttt{struct node}, that contains
pointers to the left and right sub-trees and to the parent tree. These
structs are allocated all at once, at the beginning of the algorithm. They
are kept in an array that contains exactly $|V|$ elements, each one
corresponding to a node $u \in V$. We also keep a pointer to the
\texttt{root} of the splay tree. If a given \texttt{struct} is pointed to
by the \texttt{root} pointer or has a non \texttt{NULL} parent pointer then
it is part of the splay tree. Therefore the corresponding node is in
$V'$. Otherwise the \texttt{struct} is allocated, but is not part of the
splay tree. Therefore the corresponding node is not in $V'$. Whenever the
\texttt{struct}, corresponding to node $u$, is in the splay tree, we splay
it to the root of the tree and use the size of the resulting left sub-tree
to determine its position in the topological order. Note that this
representation waists a bit of space, since full \texttt{struct node} space
is used even for nodes that are not in the splay tree. On the other hand we
save one level of indirection that would exist in an alternative that used
an array of single pointers to these \texttt{struct node} elements, which
would be \texttt{NULL} when the node was not in the splay tree. Saving a
level of indirection usually yields better cache performance and the space
penalty is this case is not too steep, particularly when $V'$ is large.

Hence we can determine $\min(v)$ by computing the minimum of all the
position values obtained in the procedure we just described. Likewise we
need to compute a similar process to obtain the value $\max(v)$. This time
we need to determine the nodes $u \in V'$ such that $(u, v) \in E_D$. The
process is essentially similar to the one we described for $\min(v)$,
except that this time $v$ is a neighbor of $u$ and not the other way
around. This means that we also must keep and adjacency list representation
of the transposed graph $G^T$ in memory. This representation is created
when the algorithm starts, along with the adjacency list representation of
the original graph $G_D$. Hence, during a verification procedure this
information is readily available.

Note that when $\max(v) < \min(v)$ there is no alternative to the previous
procedure. Therefore there is little opportunity to obtain a better bound
than $O(V \log V)$ time or more precisely $O(d(v) \log V)$, where $d(v)$ is
the degree of $v$, obtained by summing both the in-degree and the
out-degree. However for must verification procedures the resulting
situation is $\min(v) \leq \max(v)$. In these cases we need to determine
the set $I(v)$ described in Section~\ref{sec:my-idea}. This is straight
forward to obtain by first computing $\max(v)$ and then computing
$\min(v)$. The computation of $\min(v)$, iterates over all the nodes
$u \in V'$ such that an edge $(v, u)$ exists. For each such node we
determine its position $s(u)$. To determine $\min(v)$ we compute the
minimum among the $s(u)$ values. Since $\max(v)$ is already known it is
possible to determine which of the nodes $u$ are actually in
$I(v)$. Whenever a node $u \in V'$ is considered in the computation of
$\min(v)$, because an edge $(v, u)$ exists, we also check if
$s(u) \leq \max(v)$. If this condition is verified we have that
$u \in I(v)$. Hence we can determine $I(v)$ in the same $O(d(v) \log V)$
time bound.

Since this second situation is very frequent it is worthwhile to try to
improve it. Notice that computing a large set $I(v)$ only to have it be
rejected by the simulated Annealing process amounts to redundant work. The
annealing decision depends only on the size of $I(v)$. At each step the
simulated annealing algorithm determines a value $\Delta E$, which depends
on the current temperature and a random value number, chosen uniformly from
$[0,1]$. Whenever $\Delta E \leq 1 - |I(v)|$ the transition is accepted,
otherwise it is rejected. Hence accepted transitions usually occur for
negative values of $\Delta E$, in fact, the largest value of $\Delta E$ that
is significant for this problem is $1$. Any other positive values of
$\Delta E$ result in rejected transitions. Therefore our generation
procedure never generates positive values larger than $1$. Generating the
value $\Delta E$ is fast. Therefore we generate the $\Delta E$ value before
computing the set $I(v)$. This has the advantage that we do not need to
compute the complete set $I(v)$. As soon as we identify enough elements in
$I(v)$ so that the necessary inequality is no longer valid we can abort the
identification procedure and the corresponding transition. Hence our goal
is to obtain an $O((1 - \Delta E) \log V)$ time bound for each verification
process that fails. Our method is not guaranteed to obtain this bound all
the time. We improve the best case time to this bound and show that our
solution is effective in practice.

Our technique involves several nuances, which we will now explain and
simultaneously obtain more general selections of incompatible sets
$I(v)$. First we store the necessary nodes in BSTs, sorted by their
position in the topological order. This makes it easy and efficient to
obtain the $\min(v)$ and $\max(v)$ values, or their current best
approximations. Precisely we consider the set of nodes
$\out(v) = \{ u \in V' | (v, u) \in E_D\}$ and
$\inn(v) = \{ u \in V' | (u, v) \in E_D\}$, which can be used to obtain the
$\min(v)$ and $\max(v)$ values respectively. Consider the middle example in
Figure~\ref{fig:selection}. In this case we have $\out(v) = \{2, 3, 6\}$
and $\inn(v) = \{ 0, 4\}$, yielding $\min(v) = 2$ and $\max(v) = 4$
respectively. As before if we have $\max(v) = 4$ computed first then
identifying $I(v)$ can be obtained by searching for $4$ in the BST that
contains the representation of $\out(v)$. This allows us to divide the set
$\out(v)$ in two, the set $I(v) = \{ 2, 3 \}$ of elements that are smaller
than or equal to $4$ and set $\{6 \}$ which contains safe nodes larger than
$4$. This computation is supported in BSTs, either by simply counting the
number of elements less than or equal to $4$, or by actually splitting the
tree in two. We also use splay trees to store the $\inn(v)$ and $\out(v)$
sets, which provide a simple implementation of both these operations.

An important optimization that we use is the following. Assume that we know
that $\max(v) = 4$, and that we are in a situation where the simulated
annealing as decided that $\Delta E = 0$. In this case the transition will
be rejected because $\Delta E = 0 > 1-2 = 1 - |I(v)|$. Hence the order in
which we process $\out(v)$ matters. If we obtain $\out(v)$ as $6, 3, 2$
then only after we receive the number $2$ can we reject the transition. On
the other hand if we receive $\out(v)$ as $2, 3, 6$ we can reject the
transition immediately after receiving the number $3$ and therefore there
is no need to process the number $6$. In this case it is only one number
but in general this approach reduces the number of vertexes processed from
$O(d(v))$ to $O(2 - \Delta E)$. For this early termination to be effective
we need two conditions. We need to search for $\max(v)$ in the intermediate
configuration of $\out(v)$, i.e., this search is executed every time a new
node $u$ of $\out(v)$ is identified. We need the nodes in $\out(v)$ to be
obtained according to the topological order of the DAG. This second
condition amounts to keeping the adjacency list of a node $v$ sorted
according to the topological order. In general we can not guarantee this
requirement, since this order potentially changes with every step of the
annealing algorithm and updating all the corresponding adjacency lists
would amount to more work than what this optimization saves. Hence our
approach is simple and pragmatic. Each time a verification operation is
executed in node $v$ a prefix of its adjacency list is traversed, namely to
obtain $\out(v)$. This prefix could be the hole list. At the end of the
verification procedure this prefix of the adjacency list is sorted,
according to the topological order, and is used to overwrite the current
configuration of this prefix in the adjacency list. Hence at the end of
each verification procedure the corresponding prefix of the list is
guaranteed to be in the correct order. This optimization guarantees that in
a subsequent verification of node $v$ the corresponding adjacency list is
obtained according to the topological order. This reduces the amount of
nodes that are processed in the second verification from $O(d(v))$ to
$O(2 - \Delta E)$, provided the topological order has not changed much
meanwhile and that the second $\Delta E$ does not imply scanning a larger
prefix. Hence this optimization is only partially effective. In practice it
does improve performance as will be shown in Section~\ref{}. Note that the
adjacency list of $v$ will contain nodes $u$ that are not in $V'$, i.e.,
there are nodes $u \in V \setminus V'$ for which edges $(v, u) \in E_D$
exist. These nodes are placed at the end of the prefix reordering we
described. Also note that an inorder traversal of the BST containing the
partial configuration of $\out(v)$ provides the nodes in the topological
ordering which is necessary for the prefix overwrite. Hence no extra
sorting is actually necessary. Thus far in our explanation we assumed that
the whole set $\inn(v)$ is processed before the set $\out(v)$ is
considered. This is not actually the case, as such a procedure would not
actually obtain $O((2 - \Delta E) \log V)$ time, because of its dependency
on the in-degree of $v$. Note that to obtain this bound we also need to
reorganize a prefix of the adjacency list of the transposed graph
$G^T$. This time to keep it in reverse topological order. This is, likewise,
done by using the reverse inorder sequence of the BST storing the $\inn(v)$
set.

We finish this section by describing the general procedure for the choice
of $I(v)$, which is actually used in our implementation. Consider again the
situation in the middle graph of Figure~\ref{fig:selection}. In case the
value $\Delta E$ is $0$ the set $I(v) = \{ 2, 3 \}$ is rejected by the
verification procedure, because it contains two elements and the maximum
size allowed is size is one. However this rejection is excessive. We could
have chosen $I(v) = \{ 4 \}$, which contains only one element and therefore
would be accepted by the verification procedure. Recall that in this
example we have $\out(v) = \{2, 3, 6\}$ and $\inn(v) = \{ 0, 4\}$. This
means that we need to consider a wider range of incompatibility sets. Given
an integer $k$ we consider the set $I(v, k)$, which consists of the nodes
$u \in \out(v)$ such that $s(u) < k$ and the nodes $u \in \inn(v)$ such
that $k \leq s(u)$. Recall that in our exposition the nodes are being
identified by their $s(u)$ values. Hence we have that for our running
example $I(v, 7) = \{ 2, 3, 6\}$, where these nodes are obtained from
$\out(v)$. The incompatibility set we have considered so far is
$I(v, \max(v)+1) = I(v, 5) = \{2, 3 \}$. Consider instead the set
$I(v, 2) = \{ 4 \}$. This set contains only elements from $\inn(v)$ and is
smaller. Clearly we could also have considered the sets
$I(v, 4) = \{ 2, 3, 4\}$ and $I(v, 3) = \{ 2, 4\}$, which actually contain
vertexes from both sets $\out(v)$ and $\inn(v)$. However these
incompatibility sets have more than one element. A schematic representation
of these sets is shown in Figure~\ref{fig:Isets}.

\begin{figure}[tb]
  \begin{center}
    %
    \begin{pspicture}(10.0,4)
      %
      \rput[bl](1.0,1.0){
        \psmatrix[mnode=circle,colsep=0.8,rowsep=0.5]
        0 & &   &   & 4  \\
          & & 2 & 3 &   &   & 6
        \endpsmatrix
        \ncbox*[linecolor=black!15,linearc=0.3]{1,1}{1,5}
        \ncbox*[linecolor=black!15,linearc=0.3]{2,3}{2,7}
        \nput[labelsep=0.5]{180}{1,1}{$\inn(v)$}
        \nput[labelsep=0.5]{0}{2,7}{$\out(v)$}
      }
      \rput[bl](1.0,1.0){
        \psset{fillstyle=solid}
        \psmatrix[mnode=circle,colsep=0.8,rowsep=0.5]
        0 & &   &   & 4  \\
        & & 2 & 3 &   &   & 6
        \endpsmatrix
      }
      \psset{arrowsize=0.15}
      \psset{linewidth=0.01}
      \rput(! 0.0 7.7 mul 7.0 div 1.0 add 1.0){
        \rput(0.0,2.5){$I(v,0)$}
        \rput(0.0,-0.3){\psline(0.0, 2.6)}
        \rput(0.0,1.35){\psline{->}(0.1,0.0)}
        \rput(0.0,0.3){\psline{->}(-0.1,0.0)}
      }
      %
      \rput(! 1.0 7.7 mul 7.0 div 1.0 add 1.0){
        \rput[t](0.0,-0.4){$I(v,1)$}
        \rput(0.0,-0.3){\psline(0.0, 2.6)}
        \rput(0.0,1.35){\psline{->}(0.1,0.0)}
        \rput(0.0,0.3){\psline{->}(-0.1,0.0)}
      }
      %
      \rput(! 2.0 7.7 mul 7.0 div 0.9 add 1.0){
        \rput(0.0,2.5){$I(v,2)$}
        \rput(0.0,-0.3){\psline(0.0, 2.6)}
        \rput(0.0,1.35){\psline{->}(0.1,0.0)}
        \rput(0.0,0.3){\psline{->}(-0.1,0.0)}
      }
      %
      \rput(! 3.0 7.7 mul 7.0 div 1.0 add 1.0){
        \rput[t](0.0,-0.4){$I(v,3)$}
        \rput(0.0,-0.3){\psline(0.0, 2.6)}
        \rput(0.0,1.35){\psline{->}(0.1,0.0)}
        \rput(0.0,0.3){\psline{->}(-0.1,0.0)}
      }
      %
      \rput(! 4.0 7.7 mul 7.0 div 1.0 add 1.0){
        \rput(0.0,2.5){$I(v,4)$}
        \rput(0.0,-0.3){\psline(0.0, 2.6)}
        \rput(0.0,1.35){\psline{->}(0.1,0.0)}
        \rput(0.0,0.3){\psline{->}(-0.1,0.0)}
      }
      %
      \rput(! 5.0 7.7 mul 7.0 div 1.2 add 1.0){
        \rput[t](0.0,-0.4){$I(v,5)$}
        \rput(0.0,-0.3){\psline(0.0, 2.6)}
        \rput(0.0,1.35){\psline{->}(0.1,0.0)}
        \rput(0.0,0.3){\psline{->}(-0.1,0.0)}
      }
      %
      \rput(! 7.0 7.7 mul 7.0 div 1.2 add 1.0){
        \rput[t](0.0,-0.4){$I(v,7)$}
        \rput(0.0,-0.3){\psline(0.0, 2.6)}
        \rput(0.0,1.35){\psline{->}(0.1,0.0)}
        \rput(0.0,0.3){\psline{->}(-0.1,0.0)}
      }
      %
    \end{pspicture}
    %
  \end{center}
  \caption{Schematic representation of incompatibility sets.}
  \label{fig:Isets}
\end{figure}

We need a general procedure to determine if there is some set $I(v, k)$
such that $\Delta E \leq 1 - |I(v, k)|$. Our procedure works by
interleaving the iteration of the adjacency lists of $v$ in $G_D$ and in
$G^T$, i.e., we alternate between considering the nodes $u$ because there
is an outward edge of $v$, $(v, u) \in E_D$, and because there is an inward
edge into $v$, $(u, v) \in E_D$. Whenever appropriate these vertexes are
added to the corresponding BST, the one for $\out(v)$ and the one for
$\inn(v)$ respectively. In general the search proceeds while there is at
least one value of $k$ for which $\Delta E \leq 1 - |I(v, k)|$. Note that
we can compute the value $|I(v, k)|$ in $O(\log V)$ amortized time using
splay trees. Hence the verification procedure keeps track of an interval
$[a,b]$, such that any value of $k$ for which the desired condition holds
must belong to this interval. This interval is initialized with $a = 0$ and
$b = |V|$. This matches the fact that initially the BSTs are empty and
therefore all current $I(v, k)$ are also empty. Whenever the process
considers a node $u \in V'$ that is a neighbor of $v$, i.e.,
$(v, u) \in E_D$, then the value of $b$ may need to be decreased. For
example suppose that $\Delta E = 0$ in the middle graph of
Figure~\ref{fig:selection} that we are currently considering. Assume that
we first consider the nodes $3$ and $4$, which are added to the BSTs
corresponding to $\out(v)$ and $\inn(v)$ respectively. At this point the
values of $a$ and $b$ are still the initial values. Therefore the
corresponding interval is still $[0, 7]$. Now assume that the algorithm
needs to process the node $u = 2$. At this point the set $I(v, 7)$
contains, at least, the vertexes $2$ and $3$. Therefore the value of $b$
needs to decrease. To decrease this value we consider all the currently
known positions of $\out(v)$. In this case we have $\{ 2, 3\}$. Therefore
$b$ is reduced to $3$. The number of elements known to be inside $I(v, 4)$
is computed. This value is two, therefore we need to further reduce $b$ to
$2$. The number of elements known to be inside $I(v, 2)$ is one. Hence
$b = 2$. Now assume that we need to process the node $u = 0$. After this
new node is added to $\inn(v)$ the number of elements known to be in
$I(v,a) = I(v, 0)$ is at least two. Therefore we need to increase the value
of $a$. Increasing an index is slightly different from decreasing, because
this time we consider the set of known positions in $\inn(v)$ increased by
$1$, i.e., in this example $\{1, 5\}$. Hence we increase $a$ to $1$. Again
we verify that the set $I(v, 1)$ is known to have at most one element and
therefore $a$ is kept at value $1$. Hence at this point our current search
interval $[a,b]$ is $[1,2]$. The search can still proceed because this
interval is not empty. Hence assume that the search now processes node
$u = 6$. This node is added to the BST corresponding to $\out(v)$, but has
no impact in the value of $I(v, 2)$. Hence the value $b$ is kept
unaltered. An illustration of the procedure we just described is shown in
Figure~\ref{fig:Ireduction}. The top part shows the reduction of position
$b$ and the bottom part the increase of position $a$.

At this point there are no more nodes to consider. Therefore the search
terminates. The search ended with $1 = a \leq b = 2$. Hence the
verification process accepts the transition and node $v$ will be added to
$V'$. However an incompatibility set must first be removed from $V'$. We
can only use $I(v, 1)$ or $I(v, 2)$, in general we can use any integer in
$[a,b]$. To maximize the size of the sub-DAG it is best to minimize the
number of nodes that need to be removed. Hence we search for some integer
$k \in [a,b]$ such that $|I(v,k)|$ is as small as possible. We use the
increasing operation described in the previous paragraph. Therefore this
scan requires $O(d(v) \log V)$ time instead of $O((1+b-a) \log V)$. On the
other hand whenever the interval $[a,b]$ becomes empty during the
verification procedure, because at some point we have $b < a$, the
verification procedure terminates immediately by rejecting the
transition. When all the known nodes in $\in(v)$ are greater than all the
known nodes in $\out(v)$ it is only necessary to process $O(2 -\Delta E)$
nodes to reach this rejecting verdict. Hence this is the reason our
optimization technique tries to maintain the adjacency lists of $G$ sorted
according to the topological order and the adjacency lists of $G^T$ in
reverse topological order.

\begin{figure}[tb]
  \begin{center}
    %
    \begin{pspicture}(10.0,8.0)
      %
      \rput[bl](0.0,4.0){
        \rput[bl](1.0,1.0){
          \psmatrix[mnode=circle,colsep=0.8,rowsep=0.5]
          0 & &   &   & 4  \\
          & & 2 & 3 &   &   & 6
          \endpsmatrix
          \ncbox*[linecolor=black!15,linearc=0.3]{1,1}{1,5}
          \ncbox*[linecolor=black!15,linearc=0.3]{2,3}{2,7}
          \nput[labelsep=0.5]{180}{1,1}{$\inn(v)$}
          \nput[labelsep=0.5]{0}{2,7}{$\out(v)$}
        }
        \rput[bl](1.55,1.0){
          \psset{fillstyle=solid}
          \psmatrix[mnode=circle,colsep=0.8,rowsep=0.5]
          & &   &   & 4  \\
          & & 2 & 3 &   &   &
          \endpsmatrix
        }
        \psset{arrowsize=0.15}
        \psset{linewidth=0.01}
        %
        \rput(! 2.0 7.7 mul 7.0 div 0.9 add 1.0){
          \rput(0.0,2.5){$I(v,2)$}
          \rput(0.0,-0.3){\psline(0.0, 2.6)}
          \rput(0.0,1.35){\psline{->}(0.1,0.0)}
          \rput(0.0,0.3){\psline{->}(-0.1,0.0)}
        }
        %
        \rput(! 3.0 7.7 mul 7.0 div 1.0 add 1.0){
          \rput[t](0.0,-0.4){$I(v,3)$}
          \rput(0.0,-0.3){\psline(0.0, 2.6)}
          \rput(0.0,1.35){\psline{->}(0.1,0.0)}
          \rput(0.0,0.3){\psline{->}(-0.1,0.0)}
        }
        %
        \rput(! 7.0 7.7 mul 7.0 div 1.2 add 1.0){
          \rput[t](0.0,-0.4){$I(v,7)$}
          \rput(0.0,-0.3){\psline(0.0, 2.6)}
          \rput(0.0,1.35){\psline{->}(0.1,0.0)}
          \rput(0.0,0.3){\psline{->}(-0.1,0.0)}
        }
      }
      %
      \rput[bl](0.0,0.0){
        \rput[bl](1.0,1.0){
          \psmatrix[mnode=circle,colsep=0.8,rowsep=0.5]
          0 & &   &   & 4  \\
          & & 2 & 3 &   &   & 6
          \endpsmatrix
          \ncbox*[linecolor=black!15,linearc=0.3]{1,1}{1,5}
          \ncbox*[linecolor=black!15,linearc=0.3]{2,3}{2,7}
          \nput[labelsep=0.5]{180}{1,1}{$\inn(v)$}
          \nput[labelsep=0.5]{0}{2,7}{$\out(v)$}
        }
        \rput[bl](1.0,1.0){
          \psset{fillstyle=solid}
          \psmatrix[mnode=circle,colsep=0.8,rowsep=0.5]
          0 & &   &   & 4  \\
          & & 2 & 3 &   &   &
          \endpsmatrix
        }
        \psset{arrowsize=0.15}
        \psset{linewidth=0.01}
        \rput(! 0.0 7.7 mul 7.0 div 1.0 add 1.0){
          \rput(0.0,2.5){$I(v,0)$}
          \rput(0.0,-0.3){\psline(0.0, 2.6)}
          \rput(0.0,1.35){\psline{->}(0.1,0.0)}
          \rput(0.0,0.3){\psline{->}(-0.1,0.0)}
        }
        %
        \rput(! 1.0 7.7 mul 7.0 div 1.0 add 1.0){
          \rput[t](0.0,-0.4){$I(v,1)$}
          \rput(0.0,-0.3){\psline(0.0, 2.6)}
          \rput(0.0,1.35){\psline{->}(0.1,0.0)}
          \rput(0.0,0.3){\psline{->}(-0.1,0.0)}
        }
        %
      }
      %
    \end{pspicture}
    %
  \end{center}
  \caption{Schematic representation of range reduction. The top shows how
    to reduce value $b$ and the bottom how to increase value $a$.}
  \label{fig:Ireduction}
\end{figure}
%

\subsection{Optimizing Initial State}
\label{sec:optim-init-state}
%
In this section we present of some choices used to optimize the initial
state of the annealing process. The first technique is to separate the
initial graph into strongly connected components. In particular we used the
algorithm by~\citet*{Tarjan_1972}. Other efficient algorithms exist
see~\citet*{Sharir_1981} and~\citet*{cormen2009introduction}. This requires
only linear time, $O(V+E)$. This process discards all the edges across
components and keeps only the edges where both vertexes belong to the same
component. Also we use this algorithm to sort the components into
topological order, of the component graph, thereby ensuring that the
combination of the individual solutions is a global topological order. An
example of this procedure is show in Figure~\ref{fig:isolation}. The bottom
graph contains only edges inside a component and the components are in
topological order from left to right.

\begin{figure}[tb]
  \begin{center}
    \begin{pspicture}(0.0,-4.0)(7.5,2.5)
      %
      \rput[bl](0,0){
        \psframe[framearc=0.5](-0.3,-0.3)(7.8,3.0)
        % 
        \rput[bl](0,0){
          \psmatrix[mnode=circle,colsep=0.8,rowsep=0.5]
          & a \\
          b & c & d & e & f & g \\
          &   & h & i & j
          \endpsmatrix
          % 
          \psset{arrowsize=0.15}
          % 

          % SCC
          \ncline{->}{1,2}{2,2}
          \ncline{->}{2,2}{2,1}
          \ncline{->}{2,1}{1,2}

          % SCC
          \ncline{->}{2,3}{2,4}
          \ncline{->}{2,4}{3,3}
          \ncline{->}{3,3}{2,3}
          \ncline{->}{3,3}{3,4}
          \ncline{->}{3,4}{2,4}

          \psset{arcangle=30}
          % SCC
          \ncarc{->}{2,5}{3,5}
          \ncarc{->}{3,5}{2,5}

          % Inter connections
          \ncline{->}{1,2}{2,3}
          \ncline{->}{1,2}{2,5}

          \ncline{->}{2,2}{2,3}

          \ncline{->}{2,4}{2,5}
          \ncline{->}{3,4}{3,5}

          \ncline{->}{2,5}{2,6}
          % 
        }
      }
      %
      % Now the solution
      %
      \rput[bl](0,-3.5){
        \psframe[framearc=0.5](-0.3,-0.3)(7.8,3.0)
        % 
        \rput[bl](0,0){
          \psmatrix[mnode=circle,colsep=0.8,rowsep=0.5]
          & a \\
          b & c & d & e & f& g \\
          &   & h & i & j
          \endpsmatrix
          % 
          \psset{arrowsize=0.15}
          % 

          % SCC
          \ncline{->}{1,2}{2,2}
          \ncline{->}{2,2}{2,1}
          \ncline{->}{2,1}{1,2}

          % SCC
          \ncline{->}{2,3}{2,4}
          \ncline{->}{2,4}{3,3}
          \ncline{->}{3,3}{2,3}
          \ncline{->}{3,3}{3,4}
          \ncline{->}{3,4}{2,4}

          \psset{arcangle=30}
          % SCC
          \ncarc{->}{2,5}{3,5}
          \ncarc{->}{3,5}{2,5}

          % Inter connections
        }
      }
    \end{pspicture}
  \end{center}
  %
  \caption{Illustration of SCC isolation.}
  \label{fig:isolation}
\end{figure}
%

The second approach is to use an approximation algorithm. In
Section~\ref{sec:complexity-analysis} we showed that the max-DAG problem is
Poly-APX-complete, see~\ref{cor:APX}. This fact results from a reduction of
the Independent Set problem. Therefore we are not focusing on an
approximation algorithm for the max-DAG, but for its complement. The
complement of an Independent Set is a Vertex Cover. Polynomial time
algorithms that obtain a $2$ approximation are known,
see~\citet*{cormen2009introduction} or~\citet*{garey1979computers}
or~\citet*{papadimitriou1998combinatorial}. We use this algorithm to
pick an initial state for our annealing algorithm.

Given a graph $G_D = (V, E_D)$ and an ordering of the vertexes
$s:V' \mapsto \{0, \ldots, |V'|-1\}$ we can classify edges as forward or
backward. An edge $(u, v) \in E_D$ is a backward edge iff $s(u) \geq
s(v)$. Otherwise it is a forward edge. Recall that $s$ is a
bijection. Notice that this classification is not an intrinsic of an edge,
because it is influence by the ordering. Different orderings provide
different classifications.

For our objectives the important consequence of this classification is that
the graph $G_D$ is a DAG iff there is some ordering such that no edge is
backward. Hence our approximation approach consists in removing vertexes
from $G_D$ so that the resulting graph has no backward edges.

Consider the set of backwards edges $B_s \subseteq E_D$ and let $V''$ be a
vertex cover of the graph composed of these edges, i.e., the graph given by
$(V, B_s)$. Note that the notion of vertex cover is usually defined on
undirected graphs so assume that after we classify the edges its direction
is ignored. Precisely, by a set cover we mean a set of vertexes $V''$ such
that for any backward edge $(u, v) \in B_s$ at least one of the vertexes
must be in $V''$, i.e., $u \in V''$ or $v \in V''$.

Our observation is that the sub-graph induced by $V \setminus V''$ is a
DAG. This can easily be established. If this graph contained a cycle then
at least some edge $(u,v)$ in that cycle would be a backward edge and
therefore either $u$ or $v$ would be in $V''$, which would cut the
cycle.

We use a $2$ approximation algorithm to obtain $V''$ in $O(V+E)$
time. Still this does not guarantee that we obtain a $2$ complement
approximation of the max-DAG problem. The reason being that the vertex
cover we obtain depends on $s$. Hence to obtain a $2$ complement
approximation we would need to check all possible such orderings, which is
impractical. Instead we use the ordering provided by Tarjan's
algorithm. Our experimental results show that the resulting algorithm is
usually very far from a $2$ complement approximation, but the resulting
max-DAG approximation is very good.

We illustrate this result with a torus grid graph. 

\section{Related work}
\label{sec:related-work}
%(1-2 pages, 10 readers)

\section{Conclusions and further work}
\label{sec:concl-furth-work}
%(0.5 pages)

\section{Acknowledgments}
\label{sec:acknowledgements}
%
The work reported in this article was supported by national funds through
Funda\c{c}\~{a}o para a Ci\^{e}ncia e a Tecnologia (FCT) with reference
UIDB/50021/2020 and project NGPHYLO PTDC/CCI-BIO/29676/2017.
%
\bibliography{biblio}
\end{document}

%%% Local Variables:
%%% mode: latex
%%% TeX-master: t
%%% End:
{